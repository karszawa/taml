\documentclass[a4paper,11pt]{jsarticle}
\usepackage{amssymb,amsmath,graphicx, url, here}
\usepackage[top=25truemm,bottom=25truemm,left=25truemm,right=25truemm]{geometry}

\makeatletter
 \DeclareRobustCommand\cite{\unskip
\@ifnextchar[{\@tempswatrue\@citex}{\@tempswafalse\@citex[]}}
 \def\@cite#1#2{$^{\hbox{\scriptsize{[#1\if@tempswa , #2\fi]}}}$}
 \def\@biblabel#1{[#1]}
\makeatother

\begin{document}

平成27年度 第5学年 情報工学実験ⅠI

\vspace{0.5in}

\begin{center}
\Huge
外部・内部設計書

\vspace{2in}

\begin{table}[htbp]
\centering
\begin{tabular}{lll}
実験報告者 & E1110 & 唐澤弘明 \\
共同実験者 & E1031 & 西田悠 \\

\end{tabular}
\end{table}

\vspace{0.5in}

\begin{table}[htbp]
\centering
\begin{tabular}{ll}
提出日 & 2015年5月28日 \\
\end{tabular}
\end{table}
\end{center}


\setcounter{page}{0}
\thispagestyle{empty}
\newpage

\section{概要}
開発するシステムはiOSアプリとし、日々の時間割を簡単に確認でき、進級の要件にも関わる欠席数の管理を出来ることを目的とする。図\ref{img:appearance}にシステムの外観の想像図を示す。

\begin{figure}[htbp]
\centering
\includegraphics[width=60mm]{./appearance.eps}
\caption{システム外観の想像図 \label{img:appearance}}
\end{figure}


\section{システム方式}
\subsection{サーバのハードウェアとソフトウェア}
\subsubsection{サーバに関する情報}
\begin{itemize}
    \item ホスト名: navel.knet
    \item IPアドレス: 172.16.32.8
\end{itemize}

\subsubsection{サーバのハードウェア}
サーバのハードウェアにはHP Workstation上の仮想マシン(KVM)とする。


\end{document}
